\documentclass{snapshotmfo}
\categorizationmath{algebra and number theory,analysis,discrete mathematics and foundations,geometry and topology,numerics and scientific computing,probability theory and statistics}
\categorizationconnect{chemistry and earth science,computer science,engineering and technology,finance,fine arts,humanities and social sciences,life science,physics,reflections on mathematics}
\license{CC-BY-SA-4.0}
\snapshotid{373}{1950}
\junioreditor{Some One}{junior-editors@mfo.de}
\senioreditor[f]{Anja Randecker}{senior-editor@mfo.de}
\director[m]{Gerhard Huisken}
\usepackage[utf8]{inputenc}
\usepackage{amsmath,amssymb}
\usepackage{longtable}

\usepackage[spanish]{babel}
\usepackage[noabbrev,nameinlink,spanish]{cleveref}
\usepackage{ellipsis}

\author{%
narrow\texorpdfstring{\!}{}\thanks{narrow spacing}
natural\thanks{natural spacing}
wide\texorpdfstring{\,}{}\thanks{wide spacing} 
\and Author Two}
\title{spanish footnotes}
\authorinfo{\authorname{Author One} is a professor of pure mathematics at the First University.}
\authorinfo{\authorname{Author Two} is a lecturer in applied mathematics at the Second Institution.}

\begin{document}
\pdfbookmark{spanish footnotes}{snapshottitle}

\begin{abstract}
Does the spacing of the footnote marks in the title and in the text look good?
\end{abstract}

\verb+\hskip .1em+ (in snapshotmfo.cls) is a good spacing for footnote marks, to be undone by
\verb+\hskip -.1em+ (in template.tex) after punctuation marks.


\section{Case 1: after a word}

\noindent This\hskip -.1em\footnote{a} is narrow.

\noindent This\footnote{b} is natural.

\noindent This\hskip .1em\footnote{c} is wide.


\section{Case 2: between word and punctuation}

\noindent This\hskip -.1em\footnote{d}. is narrow.

\noindent This\footnote{e}. is natural.

\noindent This\hskip .1em\footnote{f}. is wide.


\subsection{with different punctuation marks}

\noindent This\footnote{g}, is natural.

\noindent This\footnote{h}; is natural.

\noindent This\footnote{i}: is natural.

\noindent This\footnote{j}? is natural.

\noindent This\footnote{k}! is natural.


\section{Case 3: after punctuation}

\noindent This.\hskip -.1em\footnote{l} is narrow.

\noindent This.\footnote{m} is natural.

\noindent This.\hskip .1em\footnote{n} is wide.


\subsection{with different punctuation marks}

\noindent This,\footnote{o} is natural.

\noindent This;\footnote{p} is natural.

\noindent This:\footnote{q} is natural.

\noindent This?\footnote{r} is natural

\noindent This!\footnote{s} is natural


\section{punctuation by itself} 

period. comma, semicolon; colon: question mark? exclamation point! and on we go

\end{document}
