\documentclass{snapshotmfo}

\categorizationmath{algebra and number theory,analysis,discrete mathematics and foundations,geometry and topology,numerics and scientific computing,probability theory and statistics} %at least one must be chosen. 
\categorizationconnect{chemistry and earth science,engineering and technology,finance,humanities and social sciences,life science,physics,reflections on mathematics} %can be void.
\license{CC-BY-SA-4.0} %recommended
\snapshotid{3}{1950}
\junioreditor{Vanessa Pi und Simon Randell}{junior-editors@mfo.de}
\senioreditor[f]{Carla Cederbaum}{senior-editor@mfo.de}
\director[m]{Gerhard Huisken}
\usepackage[utf8]{inputenc}
\usepackage{amsmath,amssymb}

%\usepackage[ngerman]{babel}
\usepackage[USenglish]{babel}

\author{Test Autor}
\title{multi-line section headings}
\begin{document}

\begin{abstract}
Compare automatic indentation to manual indentation of the second lines of multi-line headings!\end{abstract}

The indents of the respective second lines are slightly too small. As the layout settings in the snapshotmfo.cls document class are quite involved, the reason could not be found. However, you can correct the indent manually. The hspace values given in the source code of this file work for single-digit section, subsection and subsubsection numbers.

In many cases short headings will be an elegant way to circumvent this problem. 

\section[Line One ... filled up Line Two]{Line One ................................................... filled up Line Two}
Section heading with automatic line break and automatic indentation of the second line.

\section[Line One ... filled up Line Two]{Line One ................................................... filled up\\
\hspace{12.58pt}Line Two}
Section heading with manual line break and manual indentation of the second line.

\subsection[Line One ... filled up Line Two]{Line One .................................................................. filled up Line Two}
Subsection heading with automatic line break and automatic indentation of the second line.

\subsection[Line One ... filled up Line Two]{Line One .................................................................. filled up\\
\hspace{19.8pt}Line Two}
Subsection heading with manual line break and manual indentation of the second line.

\subsubsection[Line One ... filled up Line Two]{Line One ............................................................. filled up Line Two}
Subsubsection heading with automatic line break and automatic indentation of the second line.

\subsubsection[Line One ... filled up Line Two]{Line One ............................................................. filled up\\
\hspace{29.2pt}Line Two}
Subsubsection heading with manual line break and manual indentation of the second line.

\end{document}
