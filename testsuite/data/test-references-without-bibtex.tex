\documentclass{snapshotmfo}

\categorizationmath{algebra and number theory,analysis,discrete mathematics and foundations,geometry and topology,numerics and scientific computing,probability theory and statistics} %at least one must be chosen. 
\categorizationconnect{chemistry and earth science,engineering and technology,finance,humanities and social sciences,life science,physics,reflections on mathematics} %can be void.
\license{CC-BY-SA-4.0} %recommended
\snapshotid{136}{1950}
\junioreditor[m]{Some One}{junior-editors@mfo.de}
\senioreditor[f]{Carla Cederbaum}{senior-editor@mfo.de}
\director[m]{Gerhard Huisken}
\usepackage[utf8]{inputenc}
\usepackage{amsmath,amssymb}

%\usepackage[ngerman]{babel}
\usepackage[USenglish]{babel}

\author{Test Author}
\title{references without BibTeX}

%%%%%%%%%%%%%%%% BIBLIOGRAPHY %%%%%%%%%%%%%%%%%%%%%%%%%%%%%%%%%%%%%%%%%%%%%%%%%%%%%%%%%%%%%%%%%%%%%%%%%%%%
%%% There are three ways to provide your references:
%%% 
%%% [...]
%%% 
%%% 3. by using no BibTeX at all:
%%%    * delete everything from \usepackage{filecontents} until \end{filecontents}
%%%    * replace the \bibliography command at the end of this file with the
%%%      thebibliography environment containing a \bibitem command for each reference.
%%% This way is deprecated as it is less flexible than the BibTeX solutions.
%%%

\begin{document}

\begin{abstract}
The description of this test is given in the first section. The other section demonstrates this way of providing references.
\end{abstract}

\section{Comment on this unit test}
We choose the third way to provide bibliographical references as explained in template.tex:
\begin{verbatim}
There are three ways to provide your references:

[...]

3. by using no BibTeX at all:
 * delete everything from \usepackage{filecontents} until
   \end{filecontents}, 
 * replace the \bibliography command at the end of this file with
   the thebibliography environment containing a \bibitem command
   for each reference.
This way is deprecated as it is less flexible than the BibTeX
solutions.
\end{verbatim}

After two or more compilations of this file there should be no ``LaTeX'' or ``class snapshotmfo'' warnings.

\section{A heading}
Your actual snapshot.\footnote{This is a footnote.} As usual, you can give references such as \cite{knuth1984texbook, wikiMath, sample13, sample12, sample11, sample14} via the \verb+\cite+ command.\\

We appreciate if you include images or other graphics that illustrate your snapshot. However, please do keep in mind the copyright issues explained in our email in case you include images and graphics you have not produced yourself.

%%%%%%%%%%%%%%%% BIBLIOGRAPHY REVISITED %%%%%%%%%%%%%%%%%%%%%%%%%%%%%%%%%%%%%%%%%%%%%%%%%%%%%%%%%%%%%%%%%%
%%% [...]
%%% 
%%% 3. If you chose to use no BibTex at all, delete the above
%%% \bibliography command and adopt the following lines as appropriate:
\begin{thebibliography}{}
\bibitem[1]{sample14}A. Candidate, {\slshape Thesis title}, PhD thesis, MFO, 2014.
\bibitem[2]{sample11}C. Example, {\slshape A new perspective on mathematics}, New perspectives on arts and sciences, 2011.
\bibitem[3]{knuth1984texbook}D. E. Knuth, {\slshape The TeXbook}, 1st ed., Addison-Wesley, 1984.
\bibitem[4]{sample12}J.\ Sample, {\slshape Things you don't know about mathematics}, A bookseries about mathematics, Some publisher, 2012.
\bibitem[5]{sample13}\underline{\hphantom{J.\ Sam}}\,, {\slshape Interesting facts in mathematics}, arxiv:8765.4321v1, 2013.
\bibitem[6]{wikiMath}Wikipedia, {\slshape Mathematics --- {W}ikipedia{,} The Free Encyclopedia}, 2014, https://en.wikipedia.org/wiki/Mathematics, visited on May 19, 2014.
\end{thebibliography}
\end{document}
