\documentclass{snapshotmfo}

\categorizationmath{algebra and number theory,analysis,discrete mathematics and foundations,geometry and topology,numerics and scientific computing,probability theory and statistics} %at least one must be chosen.
\categorizationconnect{chemistry and earth science,engineering and technology,finance,humanities and social sciences,life science,physics,reflections on mathematics} %can be void.
\license{CC-BY-SA-4.0} %recommended
\snapshotid{272}{1950}
\junioreditor[m]{Some One}{junior-editors@mfo.de}
\senioreditor[f]{Carla Cederbaum}{senior-editor@mfo.de}
\director[m]{Gerhard Huisken}
\creditsversionone
\usepackage[utf8]{inputenc}
\usepackage{amsmath,amssymb}

\usepackage[ngerman]{babel}

\author{Test Autor}
\title{Version 1 des deutschen Abspanns}
\begin{document}
\pdfbookmark{Version 1 des deutschen Abspanns}{snapshottitle}

\begin{abstract}[Sind auf der letzten Seite dieses Schnappschusses 4 bzw. 5 Logos zu sehen, u.a. die Worte \glqq Mitglied der Leibniz-Gemeinschaft\grqq ?]
Sind auf der letzten Seite dieses Schnappschusses 4 bzw.\ 5 Logos zu sehen, u.\,a. die Worte \glqq Mitglied der Leibniz-Gemeinschaft\grqq ?
\end{abstract}

\section{Kommentar zu diesem Komponententest}
In Version 1 lautet die erste Zeile des Abspanntextes\\
\\
\hbox{\relscale{0.9}\emph{Schnappsch\"usse moderner Mathematik aus Oberwolfach} werden von Teilnehmerinnen}\\
\\
und folgende Logos werden gezeigt:
\begin{enumerate}
  \item das MFO als Mitglied der Leibniz-Gemeinschaft,
  \item die Klaus Tschira Stiftung,
  \item die Oberwolfach Foundation,
  \item Imaginary.
\end{enumerate}
Diese Version wird durch \verb+\creditsversionone+ in der Präambel dieser Datei aktiviert
und wird für Schnappschüsse verwendet, die vor dem Jahr 2017 eingereicht wurden.

\end{document}
