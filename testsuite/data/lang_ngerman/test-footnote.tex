\documentclass{snapshotmfo}
\categorizationmath{algebra and number theory,analysis,discrete mathematics and foundations,geometry and topology,numerics and scientific computing,probability theory and statistics}
\categorizationconnect{chemistry and earth science,computer science,engineering and technology,finance,fine arts,humanities and social sciences,life science,physics,reflections on mathematics}
\license{CC-BY-SA-4.0}
\snapshotid{273}{1950}
\junioreditor{Some One}{junior-editors@mfo.de}
\senioreditor[f]{Anja Randecker}{senior-editor@mfo.de}
\director[m]{Gerhard Huisken}
\usepackage[utf8]{inputenc}
\usepackage{amsmath,amssymb}
\usepackage{longtable}

\usepackage[ngerman]{babel}
\usepackage[noabbrev,nameinlink,ngerman]{cleveref}
\usepackage{ellipsis}

\author{%
natural\thanks{natural spacing} and
narrow,\texorpdfstring{\hskip -.1em}{}\thanks{narrow spacing}
spacing of footnotemarks in the title
\and Author Two}
\title{ngerman footnotes}
\authorinfo{\authorname{Author One} is a professor of pure mathematics at the First University.}
\authorinfo{\authorname{Author Two} is a lecturer in applied mathematics at the Second Institution.}

\begin{document}
\pdfbookmark{ngerman footnotes}{snapshottitle}

\begin{abstract}
Does the spacing of the footnote marks in the title and in the text look good?
\end{abstract}

In the title, use\\
\verb+\thanks+ after a word and\\
\verb+\texorpdfstring{\kern -.1em}{}\thanks{\dots}+ after a punctuation mark.

In a paragraph, use\\
\verb+\footnote+ after a punctuation mark and\\
\verb+\footnotewithspace+ after a word.


\section{Case 1: after a word}

\noindent This\footnote{a} is narrow.

\noindent This\footnotewithspace{b} is natural.


\section{Case 2: between word and punctuation}

\noindent This\footnote{d}. is narrow.

\noindent This\footnotewithspace{e}. is natural.


\subsection{with different punctuation marks}

\noindent This\footnote{g}, is narrow.

\noindent This\footnote{h}; is narrow.

\noindent This\footnote{i}: is narrow.

\noindent This\footnote{j}? is narrow.

\noindent This\footnote{k}! is narrow.


\section{Case 3: after punctuation}

\noindent This.\footnote{l} is natural.

\noindent This.\footnotewithspace{m} is wide.


\subsection{with different punctuation marks}

\noindent This,\footnote{o} is natural.

\noindent This,\footnotewithspace{o} is wide.

\noindent This;\footnote{p} is natural.

\noindent This;\footnotewithspace{p} is wide.

\noindent This:\footnote{q} is natural.

\noindent This:\footnotewithspace{q} is wide.

\noindent This?\footnote{r} is natural.

\noindent This?\footnotewithspace{r} is wide.

\noindent This!\footnote{s} is natural.

\noindent This!\footnotewithspace{s} is wide.



\section{punctuation by itself} 

period. comma, semicolon; colon: question mark? exclamation point! and on we go


\end{document}
