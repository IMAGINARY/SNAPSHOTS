%% Template for an MFO snapshot


%% Make sure that this path leads to the place where you put the class "mfosnapshot". Put the file "footer" in the same folder! 
\documentclass{mfosnapshot}


%% please choose your language! 
%\usepackage[ngerman]{babel}
\usepackage[USenglish]{babel}

%% provides demo images
\usepackage{mwe}

% mehrere Namen mit \and trennen
\author{John Thomson \and Tom Johnson}
\title{A great snapshot}

\authorinfo{\authorname{John Thomson} is professor for pure mathematics at the Honolulu University.\\\mailtoref{aloha@hangloose.us}}
\authorinfo{\authorname{Tom Johnson} is lecturer for extraterrestrial science on the moon.\\\mailtoref{ufo@manonthemoon.mo}}


% responsible organizer from your workshop
\organizer{Max Muster}
% Klassifikationsliste den Autoren zugänglich machen? 
\classification{Algebraic Geometry}
% Dieselben Daten wie in der DOI
\snapshotid{001}{2014}

\junioreditor{Lea Renner}{junior-editors@mfo.de}
\senioreditor{Dr.\;Carla Cederbaum}{cederbaum@mfo.de}

\begin{bibfilecontents}
@book{knuth1986texbook,
  keywords = {book},
  title={The texbook},
  author={Knuth, D.E. and Bibby, D.},
  volume={1993},
  year={1986},
  publisher={Addison-Wesley}
}
@article{knuth1977fast,
  keywords = {furtherreading,book},
  title={Fast pattern matching in strings},
  author={Knuth, D.E. and Morris Jr, J.H. and Pratt, V.R.},
  journal={SIAM journal on computing},
  volume={6},
  number={2},
  pages={323--350},
  year={1977},
  publisher={SIAM}
}
\end{bibfilecontents}

\junioreditor{Sophia Jahns, Lea Renner}{junior-editors@mfo.de}

\begin{document}


\begin{abstract}
This is your abstract. It should give a brief overview of your snapshot, if possible without using formulas. 
\end{abstract}

% 
\section{A heading}
Your actual snapshot. And here are some references as well: \cite{knuth1977fast} ends up in the \enquote{Further Reading} section since its \texttt{biblatex} entry contains \verb!keywords = {furtherreading}!, while \cite{knuth1986texbook} ends up in the standard references.

% Vorlage zum einbinden von Bildern 
\begin{figure}[h]
        \centering \includegraphics[height=5cm]{example-image}
        \caption{An image.}
\label{fig:moon}
\end{figure}


% Vorlage für Literaturverzeichnis? 
% Ich weiß nicht genau, wie das am Ende aussieht, deshalb hab ich keinen Vorschlag gemacht. 


\end{document}